\documentclass[a2paper, 14pt, portrait, 
					innermargin = 25mm, 
					blockverticalspace=0.25cm]{tikzposter}

\usepackage{xcolor}

% TikzPoster Presets:

	\definecolor{satpink}{RGB}{213, 53, 118}
	\definecolor{backblue}{HTML}{181A34}
	\definecolor{ppink}{HTML}{E70074}
	\definecolor{yyellow}{HTML}{FFFE03}
	\definecolor{blockblue}{HTML}{3A3F80}

	\definecolorstyle{Moscow}{
		\colorlet{One}{satpink}
		\colorlet{Two}{ppink}
		\colorlet{Three}{blockblue}
	}{
	% Background
		\colorlet{backgroundcolor}{backblue}
		\colorlet{framecolor}{backblue}
	% Title
		\colorlet{titlefgcolor}{white}
		\colorlet{titlebgcolor}{backblue}
	% Blocks
		\colorlet{blocktitlebgcolor}{satpink} 
		\colorlet{blocktitlefgcolor}{white} 
		\colorlet{blockbodybgcolor}{backblue} 
		\colorlet{blockbodyfgcolor}{white}
	% Innerblock colors 
		\colorlet{innerblocktitlebgcolor}{white} 	\colorlet{innerblocktitlefgcolor}{black} \colorlet{innerblockbodybgcolor}{white} \colorlet{innerblockbodyfgcolor}{black}
	% Note colors 
		\colorlet{notefgcolor}{black} 
		\colorlet{notebgcolor}{blockblue} 
		\colorlet{notefrcolor}{white}
	}

\usecolorstyle{Moscow}
\useblockstyle{Minimal}


% Font Presets:
	\usepackage[defaultfam,regular,tabular,lining, bold]{montserrat}
	\usepackage[T1]{fontenc}
	\renewcommand*\oldstylenums[1]{{\fontfamily{Montserrat-TLF}\selectfont #1}}
	\renewcommand*\familydefault{\sfdefault}

	\usepackage{xpatch}
	\xpatchcmd{\block}{\bf\LARGE\color{blocktitlefgcolor}}{\sffamily\LARGE\color{blocktitlefgcolor}}{}{} 


% Stuff:
	\usepackage{graphicx}
	\usepackage{tikz}
	\usepackage{pgfplots}
		\pgfplotsset{compat=1.15}
	\usepackage{pgfplotstable}
	\usepackage{fontawesome5}
	\usepackage{etoolbox}
	\usepackage{soul}
	\usepackage[colorlinks=false]{hyperref}


% Turn off numeration of Tikz captions:
	\renewenvironment{tikzfigure}[1][]{
		\def \rememberparameter{#1}
		\vspace{10pt}
		\refstepcounter{figurecounter}
		\begin{center}
		}{
			\ifx\rememberparameter\@empty
			\else %nothing
			\\[10pt]
			%{\small Fig.~\thefigurecounter: \rememberparameter}
			{\small \rememberparameter}
			\fi
		\end{center}
	}


\title{\color{yyellow} \montserratalt Moscow Sunset and Sunrise Map}
\author{\faGithub \, ivanskv2000}
%\titlegraphic{\includegraphics[width=0.2\linewidth]{logo_white.pdf}}
\date{\today}


\begin{document} 

\maketitle[titletoblockverticalspace = -0.5cm]

\block{Project Description}{
		The aim of this project was to compile a Moscow map representing whether a street is appropriate for sunrise (or sunset)
		 observation or not. Final result is based on \textbf{street azimuths} which are being compared to monthly 
		 solar data. The output is both beautiful and functional, athough the latter still needs to be proved.
} 

\block{What's Under the Hood?}{
	The ``artistic path'' of this project consists of several steps: data retrieval, data processing, 
	map compilation and post-processing. Rough geospatial data is provided by \textbf{OpenStreetMap} community; 
	solar data is computed by employing the \textit{astral} Python package. Another source is \textit{Elevation API} 
	used for additional investigation, which is not included into the visualisations. A set 
	of Python scripts hosted on GitHub provides a straightforward way of building the map on your own 
	(and a precise description of the compilation process can be found in Jupyter notebook).
} 

\block{Examples}{
	\begin{center}
		\begin{minipage}{0.45\linewidth}
			\centering
			\begin{tikzfigure}[\color{blockblue}\montserratalt\large
					January
				]
				\includegraphics[width=\linewidth]{map_January.pdf}
			\end{tikzfigure}%
		\end{minipage}\hfill
		\begin{minipage}{0.45\linewidth}
			\centering
			\begin{tikzfigure}[\color{blockblue}\montserratalt\large
					July
				]
				\includegraphics[width=\linewidth]{map_July.pdf}
			\end{tikzfigure}%
		\end{minipage}\hfill
	\end{center}
} 

\begin{columns} 
	\column{0.5}
		\block{Python packages}{
		\begin{itemize}
			\item Boeing, G. 2017. \textit{``OSMnx: New Methods for Acquiring, Constructing, Analyzing, and Visualizing Complex Street Networks.''} Computers, Environment and Urban Systems. 65, 126-139. \href{https://doi.org/10.1016/j.compenvurbsys.2017.05.004}{doi:10.1016/j.compenvurbsys.2017.05.004}
			\item Kennedy, S. (n. d.) \textit{Astral Python package.}  \href{https://github.com/sffjunkie/astral}{github.com/sffjunkie/astral}
		\end{itemize} 
	}
	\column{0.5} 
		\block{Spatial data sources}{
		\begin{itemize}
			\item OpenStreetMap contributors (2020). \textit{Moscow geospatial data.} \href{https://www.openstreetmap.org}{openstreetmap.org} 
			\item \textit{Elevation api.} (n. d.) \href{https://elevation-api.io}{elevation-api.io}
		\end{itemize} 
	}
\end{columns}

\end{document}
